%!TEX root = main_acm.tex

\section{Conclusion}
\label{sec:con}

We presented a passive method to study IP anycast by utilizing BGP data.  We
proposed a set of BGP-related features (thus not based on active measurements)
to classify anycast and unicast prefixes. Extracting data from RouteViews and
RIPE RIS, we evaluated the effectiveness of our proposed approach against a
near-ground-truth dataset based on active-probing
measurements~\cite{cicalese2015characterizing}. The evaluation results show that
our approach achieves high classification accuracy---about 90\% for anycast and
99\% for unicast---and is also able to detect anycast prefixes incorrectly
labeled as unicast in the near-ground-truth dataset. 

In addition, while delving into the causes of inaccuracy, we found indication
that remote peering might have an unintended impact on anycast routing. We
investigated this phenomenon by combining regular traceroutes, measurements
executed with the traIXroute~\cite{traixroute, traixroute:pam} open-source tool,
BGP data from RouteViews and RIPE RIS, and data from the Remote IXP Peering
Observatory \cite{RPJedi}.  Our study showed that remote peering has the
potential to affect 19.2\% of the anycast prefixes and we confirmed via
traceroute measurements that around 40\% of such prefixes were indeed impacted
by remote peering.  We also revealed that remote peering could increase
transmission latency by routing traffic to distant suboptimal anycast sites.

